\documentclass[lualatex,a5paper,ja=standard]{bxjsarticle}
\usepackage[utf8]{inputenc}
\usepackage{enumitem}
\usepackage{tikz}
\usepackage{amsmath}
\usepackage{amssymb}
\usepackage{luacode}
\usetikzlibrary{intersections, calc, angles, quotes}


\title{壁と物体の跳ね返りの話}
\author{Torajiro Aida}

\begin{document}

\maketitle


物体1の質量を$1$, 物体2の質量を$m$とする. $m \ge 1$とする.

物体1, 物体2の速度(の数列)をそれぞれ$v_1$, $v_2$とする.

物体1, 物体2の初期速度をそれぞれ$0$, $V$とする($V$は正).

座標が大きい順に, 壁面, 物体1, 物体2の順に並んでいるとする. この時, 何回の衝突が発生するかを考える.

物体1と物体2の衝突前後を考える. ここで運動量保存則
\begin{equation}
    v_1 + m v_2 = v_1' + m v_2'\label{pp}
\end{equation}
と完全弾性衝突
\begin{equation}
    v_1 - v_2 = -v_1' + v_2' \label{ep}
\end{equation}を仮定する. これを解くと, 
\begin{eqnarray}
    v_1' &=& -\frac{m-1}{m+1}v_1 + \frac{2}{m+1} v_2 \\
    v_2' &=& \frac{2}{m+1} v_1 + \frac{m-1}{m+1} v_2
\end{eqnarray}
となる.

$\alpha = \frac{m-1}{m+1}$, $\beta = \frac{2}{m+1}$とする.
ここで, 物体1が物体2とぶつかり, その後, 壁にぶつかり跳ね返ったとする場合の変換
\begin{equation}
\begin{pmatrix}
    v_{1_{n+1}}\\
    v_{2_{n+1}}
\end{pmatrix}
    =
\begin{pmatrix}
    \alpha & -\beta m\\
    \beta & \alpha
\end{pmatrix}
\begin{pmatrix}
    v_{1_{n}}\\
    v_{2_{n}}
\end{pmatrix}
\end{equation}
を考える. ここで, この行列の行列式は, 
\begin{equation}
    \alpha^2 + m\beta^2 = 1  \label{det}
\end{equation}
である.

この変換を初期状態からn回繰り返すと, 
\begin{equation}
    \begin{pmatrix}
        v_{1_n}\\
        v_{2_n}
    \end{pmatrix}
        =
    \begin{pmatrix}
        \alpha & -\beta m\\
        \beta & \alpha
    \end{pmatrix}^n
    \begin{pmatrix}
        0\\
        V
    \end{pmatrix}
\end{equation}
となる. この行列を, その固有値$\lambda_1 = \alpha + \beta \sqrt{m} i$と$\lambda_2 = \alpha - \beta \sqrt{m} i$を使って対角化すると, 
\begin{equation}
\begin{pmatrix}
    \alpha & -\beta m\\
    \beta & \alpha
\end{pmatrix}=\frac{1}{2\sqrt{m}i}\begin{pmatrix}
    \sqrt{m}i & -\sqrt{m}i\\
    1 & 1
\end{pmatrix}\begin{pmatrix}
    \lambda_1 & 0 \\
    0 & \lambda_2 \\
\end{pmatrix}\begin{pmatrix}
    1 & \sqrt{m}i\\
    -1 & \sqrt{m}i
\end{pmatrix}
\end{equation}を得る. 従って, 
\begin{eqnarray}
    \begin{pmatrix}
        v_{1_n}\\
        v_{2_n}
    \end{pmatrix}
        &=&
        \frac{1}{2\sqrt{m}i}\begin{pmatrix}
            \sqrt{m}i & -\sqrt{m}i\\
            1 & 1
        \end{pmatrix}\begin{pmatrix}
            \lambda_1 & 0 \\
            0 & \lambda_2 \\
        \end{pmatrix}^n\begin{pmatrix}
            1 & \sqrt{m}i\\
            -1 & \sqrt{m}i
        \end{pmatrix}\begin{pmatrix}
            0\\
            V
        \end{pmatrix}\\
        &=&
        \frac{1}{2\sqrt{m}i}\begin{pmatrix}
            \sqrt{m}i & -\sqrt{m}i\\
            1 & 1
        \end{pmatrix}\begin{pmatrix}
            \lambda_1^n V\sqrt{m}i \\
            \lambda_2^n V\sqrt{m}i \\
        \end{pmatrix}\\
        &=&
        \frac{1}{2}V\begin{pmatrix}
            \sqrt{m}(\lambda_1^n-\lambda_2^n)i \\
            \lambda_1^n+\lambda_2^n \\
        \end{pmatrix}
\end{eqnarray}

ここで, (\ref{det})より, ある$\theta$が存在して, $\alpha = \cos \theta, \beta = \frac{1}{\sqrt{m}} \sin \theta$となる.
また, $\lambda_1 = e^{i\theta}, \lambda_2 = e^{-i\theta}$となる. よって, 
\begin{eqnarray}
    \begin{pmatrix}
        v_{1_n}\\
        v_{2_n}
    \end{pmatrix}
    &=&
    \frac{1}{2}V\begin{pmatrix}
        \sqrt{m}(2i\sin(n\theta))i \\
        2\cos(n\theta) \\
    \end{pmatrix}\\
        &=&
        \begin{pmatrix}
            -\sqrt{m}V\sin(n\theta) \\
            V\cos(n\theta) \\
        \end{pmatrix} \label{res1}
\end{eqnarray}

$m \ge 1$より, $0 < \theta \le \frac{\pi}{2}$. 物体1が物体2とn回衝突した後, もう二度と衝突しない条件は, $v_{2_n} < 0$かつ$|v_{2_n}| \ge |v_{1_n}|$.

xy座標上の点$(\cos(n\theta), \sin(n\theta))$を考える. $v_1 = -\sqrt{m}V\sin(n\theta) = -\sqrt{m}Vy$, $v_2 = V\cos(n\theta) = Vx$だから, 先程の条件は$x<0$かつ$\frac{1}{\sqrt{m}} |x| \ge |y|$と表せる.
$\frac{1}{\sqrt{m}} x = y$と$x^2 + y^2 = 1$の交点の一つは$x=\sqrt{\frac{m}{m+1}}$であるが, この2倍角に対応するx座標は, $x=\alpha$となる. 従って, $\pm\frac{1}{\sqrt{m}} x = y$の間の角度は$\theta$である.

以上を図で示すと, 以下のようになる.

\begin{luacode*}
    m=10
    xs={}
    ys={}
    th=math.acos((m-1)/(m+1))
    for n = 1, 10 do
        xs[n]=math.cos(n*th)
        ys[n]=math.sin(n*th)
    end
    line = 1.5/math.sqrt(m)

    t1x = math.cos(math.pi-th/2)
    t1y = math.sin(math.pi-th/2)
    t2x = math.cos(math.pi+th/2)
    t2y = math.sin(math.pi+th/2)

    as=180 - th * 360 / (2 * math.pi) / 2
    ae=180 + th * 360 / (2 * math.pi) / 2

    tr = th * 360 / (2 * math.pi)
\end{luacode*}

\begin{center}
    \begin{tikzpicture}[x=2cm,y=2cm]
        \coordinate (O) at (0,0);
        \draw [thick, -stealth](-1.5,0)--(1.5,0) node [anchor=north]{x};
        \draw [thick, -stealth](0,-1.5)--(0,1.5) node [anchor=east]{y};
        \draw (O) circle [radius=1];
        \coordinate (L0) at (1, 0);
        \coordinate (L1) at (\directlua{tex.sprint(xs[1])},\directlua{tex.sprint(ys[1])});
        \coordinate (L2) at (\directlua{tex.sprint(xs[2])},\directlua{tex.sprint(ys[2])});
        \coordinate (L3) at (\directlua{tex.sprint(xs[3])},\directlua{tex.sprint(ys[3])});
        \coordinate (L4) at (\directlua{tex.sprint(xs[4])},\directlua{tex.sprint(ys[4])});
        \coordinate (L5) at (\directlua{tex.sprint(xs[5])},\directlua{tex.sprint(ys[5])});
        \fill (L1) circle [radius=2pt];
        \draw (0, 0) -- (L1) node [anchor=south]{$\theta$};
        \fill (L2) circle [radius=2pt];
        \draw (0, 0) -- (L2) node [anchor=south]{$2\theta$};
        \fill (L3) circle [radius=2pt];
        \draw (0, 0) -- (L3) node [anchor=south]{$3\theta$};
        \fill (L4) circle [radius=2pt];
        \draw (0, 0) -- (L4) node [anchor=south]{$4\theta$};
        \fill (L5) circle [radius=2pt];
        \draw (0, 0) -- (L5) node [anchor=south]{$5\theta$};
        \draw [dashed] (0, 0) -- (-1.5, \directlua{tex.sprint(line)}) node [anchor=south]{$y=-\frac{1}{\sqrt{m}}x$};
        \draw [dashed] (0, 0) -- (-1.5, -\directlua{tex.sprint(line)}) node [anchor=north]{$y=\frac{1}{\sqrt{m}}x$};

        \coordinate (T1) at (\directlua{tex.sprint(t1x)}, \directlua{tex.sprint(t1y)});
        \coordinate (T2) at (\directlua{tex.sprint(t2x)}, \directlua{tex.sprint(t2y)});

        \draw pic[draw=black, "$\theta$", thick, angle eccentricity=1.2, angle radius=0.8cm] {angle=L0--O--L1};

        \fill [fill=blue, opacity=.5] (0, 0) -- (T1) arc (\directlua{tex.sprint(as)}:\directlua{tex.sprint(ae)}:1) -- cycle;
    \end{tikzpicture}
\end{center}

物体1と物体2がn回衝突した後, $v_1$と$v_2$に対応する点$(\cos(n\theta), \sin(n\theta))$が青色の領域に入っていた場合, 二度と物体1と物体2が衝突することはない.
このとき, $\sin(n\theta)>0$の場合は, 物体1はその後壁に衝突する. いっぽうで$\sin(n\theta)<0$の場合は, 壁に衝突せず, 壁から遠ざかる方向(すなわち負の方向)に等速直線運動する.

従って, 壁との衝突回数も含めた合計の衝突回数は,
$$n_{\mathrm{total}} = 2\lceil \frac{\pi - \frac{\theta}{2}}{\theta} \rceil - \begin{cases}
    1\quad\mathrm{(最後に壁に衝突しない場合)}\\
    0\quad\mathrm{(最後に壁に衝突する場合)}
\end{cases}$$
と表せる.

ここで, $m \rightarrow \infty$のとき, $\theta \rightarrow 0$であり, $\frac{2\sqrt{m}}{m+1} = \sin\theta$,  $ \sin\theta\rightarrow \theta$, $\frac{2\sqrt{m}}{m+1} \rightarrow \frac{2}{\sqrt{m}}$. したがって,
$$\frac{n_{\mathrm{total}}}{\sqrt{m}} \rightarrow \pi$$

\end{document}
