\documentclass[a4paper,12pt]{article}
\usepackage{luatexja}
\usepackage{url}
\usepackage{luacode}
\usepackage{graphicx,caption}
\usepackage[absolute]{textpos}
\usepackage{geometry}
\usepackage{environ}
\usepackage{here}
\usepackage{lastpage}
\usepackage{amsmath}
\usepackage{fontspec}
\setmainfont[Language=Japanese]{HaranoAjiMincho}

\usepackage{polyglossia}
\setdefaultlanguage{japanese}
\PolyglossiaSetup{japanese}{indentfirst=true}

\setlength{\parindent}{1em}

\title{\vspace{-2cm}タイトル}
\author{わたしのなまえかもしれない}
\date{9999年9月99日}

\begin{document}

\begin{luacode*}
refTable = {}
function getlength (a)
    return #a
end
\end{luacode*}

\newcommand{\refdef}[3]{%
\expandafter\newcommand\csname ref#1\endcsname{[#2]}%
\expandafter\newcommand\csname refp#1\endcsname[1]{[#2: ##1]}%
\expandafter\newcommand\csname refc#1\endcsname{#3}%
\directlua{table.insert(refTable, "#1")}
}

%TODO: implement automatic ordering

\refdef{SomeJapaneseBook}{山田 (9999)}{山田存在しない名前 (9999)「うおおおおおおお------がががが」『なんでもないよ』赦そう機構}

\refdef{SomeEnglishBook}{Yamada et al. (9999)}{S. Yamada, N. Tanaka, M. Dameda, S. Sonnakotonaiyo (9999) From Futa-Kill Eology to Plumification of Kowadom, \textit{Journal of Ethical Implerealism}71, Kawaiiyo Foundation}

\newcommand{\thebib}{
    \setlength{\parindent}{0pt}
    \directlua{
        for i = 1, getlength(refTable) do
            tex.sprint('\\refc' .. refTable[i] .. '\\\\\\\\')
        end
    }
}


\maketitle
\TPGrid[10mm,10mm]{1}{1}
\begin{textblock}{0.13}[1,0](1,0)
    \input{count.tex}文字
\end{textblock}

\pagenumbering{arabic}

さいしょのセクションのまえになんか導入をかいたほうが良いかもね?aaaa

\section{はじめに------豆腐と旅と}

じゅうまんぺーじもよんだのに、なんにもなくて、もんもんと\refpSomeJapaneseBook{pp. 3721696-3814141}。

しているそんな、あなたには\refSomeEnglishBook。あああああ工事中。

豆腐には、オリーブオイルとコリアンダーと塩をかけて食べるとおいしいいよ、

\section{ぴょんぴょんと跳ねる、ということ}

うんうん、そうだね

うごごごごごうごごごごごうごごごごごうごごごごごうごごごごごうごごごごごうごごごごごうごごごごごうごごごごごうごごごごごうごごごごごうごごごごごうごごごごごうごごごごごうごごごごごうごごごごごうごごごごごうごごごごごうごごごごごうごごごごごうごごごごごうごごごごごうごごごごごうごごごごごうごごごごごうごごごごごうごごごごごうごごごごごうごごごごごうごごごごごうごごごごごうごごごごごうごごごごごうごごごごごうごごごごごうごごごごごうごごごごごうごごごごごうごごごごごうごごごごごうごごごごごうごごごごごうごごごごごうごごごごごうごごごごごうごごごごごうごごごごごうごごごごごうごごごごご。

\section{ぎざぎざの彼方にあるもの}

もももももも?もも?ももも?もももももも?もも?ももも?もももももも?もも?ももも?もももももも?もも?ももも?もももももも?もも?ももも?もももももも?もも?ももも?もももももも?もも?ももも?もももももも?もも?ももも?もももももも?もも?ももも?もももももも?もも?ももも?

\section{医師免許を持つ石の意思}

\section{おわりに代えて------あるいは、宇宙の3分間クッキング}

\newpage

\section*{参考文献}

\thebib

\end{document}
